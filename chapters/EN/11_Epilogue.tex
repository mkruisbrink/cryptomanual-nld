\chapter{EPILOGUE}
\label{ch:epilogue}

When you consider that the cryptocurrency market remains highly volatile and largely unregulated, it is still highly speculative to invest in any of it. The road to mass adoption is not without significant hiccups. We need to give distribute ledger technology and blockchain technology time to mature. If given that time, all these daring and promising startups will have to prove themselves, just like any other company and any other technology. Even though there is much debate around the question how we should deal with blockchain and cryptocurrencies in the coming years, history shows us that, the first reactions to any major scientific breakthroughs were never all-out positive from day one.\medskip


\section*{Financial-Technology Sector}
Besides, this market has enormous potential when you think of the possibilities for potential growth on investments in the long term since it is still in its infancy stages. Some of these projects are here to stay, and many more are starting every day, introducing healthy competition that is going to boost the fin-tech industry. There is an ever-increasing demand for skilled people in the fin-tech sector. Businesses cannot seem to find enough software engineers and blockchain developers, and everyone can get into it - take several free courses available on the internet to get you up to speed and develop new skill-sets that might boost your career and give you a new direction - this industry is just getting started. Not to mention the fact that the younger generations have been growing up in the digital age and the impact this makes on their lives cannot be unseen.\medskip

Mainly, the success of blockchain technology has already been set in stone since it's not dependent on cryptocurrency alone. Where blockchain can serve as a decentralised ledger that securely stores immutable data, we can also use it as such without the use of any cryptocurrencies. Cryptocurrencies or tokens might be used on specific networks because they have a serve a particular function. They might function as a utility token or as a means of payment to the network or to store, capture or move value around in the form of a digital currency.\medskip


\section*{Blockchain is Here to Stay}
We believe blockchain is here to stay, and it will be an integral element in our future infrastructure. The challenge is to oversee and guide the adoption process. For example, it is almost inevitable that private blockchain infrastructures are going to be used by governments and corporations. Good or bad? That's a loaded question as there might be more than one answer. Even though it may sound bad for cryptocurrency or blockchain in general - adaptive governments and nimble and experimental regulation, legislation and compliance policies will be of crucial importance to propel development forward. 

\section*{Open Blockchain Enables Inclusion}
Cryptocurrencies, particularly Bitcoin, already have massive momentum. We are not predicting whether or not all of this will eventually succeed. However, we do believe that the economy works best when it works for everyone, and this new platform represents an engine which stimulates and enables inclusion. Open blockchain significantly lowers the requirements for people to obtain access to financial services and will compete with traditional central banking models. With Bitcoin, it is also completely different in terms of spending. Open blockchains like Bitcoins do not care; they are peer to peer. You can send it from person to person, without an intermediary. 
In summary, when you store money at the bank (or in the possible future at corporations), the bank becomes the owner of that money. With Bitcoin, \emph{you} own your money. When the banks own that money, they spend it as they wish. When you own it, \emph{you} spend it as you wish. It is censorship-resistant, and no one decides on what you can or cannot spend it.\medskip 

Besides Bitcoin, there are hundreds of other projects out there that are working on solving some of the most pressing matters in our financial spheres and other industries like supply chain, healthcare, energy and content creation. We believe blockchain is here to stay, and it will be an integral element in our future infrastructure.  The challenge is to oversee and guide the adoption process. For example, it is almost inevitable that private blockchain infrastructures are going to be used by governments and corporations.  Good or bad? That's a loaded question as there might be more than one answer.  Even though it may sound bad for cryptocurrency or blockchain in general - adaptive governments and experimental Regulation, legislation and compliance policies will be of crucial importance to propel development forward.

It often requires some (catastrophic) crisis to see what is in plain sight. When the next financial crisis hits us, the world economy will suffer a massive blow, and billions of people will not have seen it coming. It is time to prepare - because this time it is global.



\section*{Collapse or metamorphosis - a black swan event}
Every potential crisis provides an opportunity for significant upside. Whenever the unimaginable occurs - it is easy to say \say{we could have seen this coming} in hindsight. 

 \medskip
    \tcbset{colback=orange!3!white,fonttitle=\bfseries}
    \begin{tcolorbox}
    [enhanced,
    title=Black swan theory,
    frame style=
    {left color=orange!85!black,right color=yellow!95!black}]

 The black swan theory or theory of black swan events is a metaphor that describes an event that comes as a surprise, has a significant effect, and is often inappropriately rationalised after the fact with the benefit of hindsight. The term is based on an ancient saying that presumed black swans did not exist - a saying that became reinterpreted to teach a different lesson after black swans were discovered in the wild. The theory was developed by Nassim Nicholas Taleb \cite{Nassim} to explain:   

\tcblower

\footnotesize \begin{enumerate}
                    \item The disproportionate role of high-profile, hard-to-predict, and rare events that are beyond the realm of normal expectations in history, science, finance, and technology.
                    \item The non-computability of the probability of the rare consequential events using scientific methods (owing to the very nature of small probabilities).
                    \item The psychological biases that blind people, both individually and collectively, to uncertainty and a rare event's massive role in historical affairs.
                \end{enumerate}

\end{tcolorbox}
\medskip


The black swan metaphor is attractive in a way - as a new global financial crisis would perfectly fit the bill. 


\section*{Both the Economy and Currencies need to evolve}

In terms of global macroeconomic developments, and after having taken a more in-depth look at the blockchain and cryptocurrency spheres, we note many indicators which lead us to believe that rapid, drastic change may lie around the corner. We actively research these areas and look at these global developments, which are driving innovation and change. We see trends that have the potential to radically change vast chunks of industry and change the lives of billions of people and find hope and opportunity in areas where others may not. In the face of any crisis, there is \emph{always} ample room for new growth,  in directions that may come as a surprise to many. What will happen when the inevitable strikes? Will economies collapse globally? Will life as we know it, be a thing of the past? If history is any indication, governments and banks will, once again, try to change the rules. 

\begin{quotation}

  \textit{\say{In the midst of every crisis, lies great opportunity.}}
  \begin{flushright}
    \small{--- \textbf{Einstein, Albert}}
  \end{flushright}

\end{quotation}

We believe that this transition heralds the beginning of a new economic era, of an inclusive economy, where people's (intellectual) contributions are valued and rewarded according to the added value they bring to the economy as a whole, their immediate environment, community or family and friends. In the Cryptomanuals, we inform people of what is going on in the global economy, provide some historical context and point out one of the most significant opportunities and wealth transfers of our lifetimes. Even though we stress radical change appears to be imminent, most people still refuse to see what is right in front of them. 

