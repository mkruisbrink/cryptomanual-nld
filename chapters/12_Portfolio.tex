\chapter{Portfolio Management}
\label{ch:portfolio}

Portfolio management gaat over het nemen van beslissingen. In markten kun je sterke en zwakke punten, bedreigingen en kansen identificeren. Terwijl je werkt aan een evenwichtige beleggingsportefeuille, ben je voortdurend bewust van de context van je belegging door het lezen van ontwikkelingen en het uitvoeren van onderzoek. Houd er rekening mee dat je beleggingsportefeuille een weerspiegeling is van jouw beslissingen. Als je gedurfde of slecht berekende beslissingen maakt, loop je dus mogelijk een hoog risico over je gehele beleggingsportefeuille. Er zijn verschillende essenti\"ele concepten waar rekening mee moet worden gehouden bij het samenstellen van een investeringsportefeuille. 

Eerst gaan we wat dieper in op de verschillende activa waar men in kan beleggen of investeren. Een gezonde investeringsportefeuille bestaat namelijk \emph{niet} alleen uit cryptocurrencies.

    \bigskip
    \begin{cryptobox}{HIGH RISK \& HIGH REWARD}
    Sommige investeringsmogelijkheden worden beschouwd als '\emph{high risk, high reward}'. Denk bijvoorbeeld aan de winstgevendheid van het investeren in internetapplicaties en -protocollen in het verleden, zoals Apple en Facebook in de tijd toen deze nog vanuit hun garage opereerden. Denk vervolgens aan Bitcoin en de snelle ontwikkeling van de blockchain technologie. 
    \tcblower
    Achteraf gezien was dit wellicht weer een '\emph{once in a lifetime} opportunity'. Wat als we je vertellen dat er meer is?
    \end{cryptobox}


\section{Belangrijke begrippen}
Als investeerder moet je een aantal essenti\"ele beleggingsconcepten begrijpen, waaronder de evaluatie van de beleggingsprestaties, de allocatie van de activa, de diversificatie van je bezittingen, de herbalancering en de rol die risico's spelen in vrijwel alle aspecten van het beleggen.\footnote{Finra; \href{https://www.finra.org/investors/key-investing-concepts}{Key Investing Topics}.}

    \begin{quotation}
          \textit{\say{Met de juiste kennis en strategie - en een portie gezond verstand - is het mogelijk om op lange termijn gezonde winsten te behalen.}}
          \begin{flushright}
            \small{--- \textbf{cryptomanuals}}
          \end{flushright}
    \end{quotation}

\subsection{Investeringsprestaties}
Het kiezen van investeringen is slechts het begin van je werk als belegger. Naarmate de tijd vordert worden de prestaties van de beleggingen ge\"evalueerd om te zien hoe ze in de portefeuille presteren en ze bijdragen om de doelen te bereiken. Over het algemeen betekent vooruitgang dat de waarde van je portefeuille gestaag stijgt, ook al hebben \'e\'en of meer van jouw beleggingen mogelijk aan waarde ingeboet.

Omdat de beleggingsmarkten voortdurend veranderen, zul je ook alert moeten zijn op mogelijkheden om de prestaties van je portefeuille te verbeteren, bijvoorbeeld door te diversifi\"eren naar een andere sector van de economie of door een deel van je portefeuille toe te wijzen aan internationale investeringen. Om geld vrij te maken voor deze nieuwe aankopen, kunnen individuele investeringen verkocht worden die niet goed hebben gepresteerd, of juist investeringen waar winst op is gemaakt.

\subsection{Allocatie binnen de portefeuille}
Wanneer activa worden toegewezen, beslist men - meestal op een procentuele basis - welk deel van de totale portefeuille wordt belegd in verschillende activaklassen, zoals aandelen, cryptocurrencies, goud en de geldmarkten. Deze beleggingen kunnen direct worden gedaan door individuele activa te kopen of indirect door te kiezen voor fondsen die in deze activa beleggen. Naarmate je een uitgebreidere portefeuille opbouwt, kunnen ook andere activaklassen, worden opgenomen die kunnen helpen om het beleggingsrisico te spreiden en dus het risico te matigen, zie \cref{sec:activaklassen}.

    \bigskip
    \begin{tipbox}{TIP}
    Het toewijzen van activa ('asset' allocatie) is een nuttig instrument om systematisch risico's te beheren omdat verschillende categorie\"en van investeringen op diverse manieren reageren op veranderende economische en politieke omstandigheden. 
    \end{tipbox}
    \bigskip


Door verschillende activaklassen in de portefeuille op te nemen, verhoogt men de kans dat een aantal van de beleggingen een bevredigend rendement opleveren, zelfs als andere onveranderd blijven of waarde verliezen. De activiteit om het risico van de beleggingsportefeuille te verminderen door je beleggingen te spreiden over verschillende activaklassen en -subklassen wordt assetallocatie genoemd. Een voorbeeldportefeuille en enkele generieke portefeuilletoewijzingen worden weergegeven in \cref{fig:portfolio example,fig:portfolio allocation,fig:subclasses}.

\begin{figure}
    \centering
   
    \begin{tikzpicture}
        
        \pie[radius=3, text=legend]{20/edelmetaal, 20/cryptocurrencies, 20/contant (equivalenten), 10/aandelen, 10/onroerend goed, 10/fondsen, 10/andere}
   
    \end{tikzpicture}
    
    \caption{Portfolio allocatie voorbeeld}
    \label{fig:portfolio example}
\end{figure}

\begin{figure}
    \centering
   
    \begin{tikzpicture}
        \pie [pos={0,0}, radius=2]{80/,20/}
        \pie [pos={4.5,0}, radius=2]{40/,30/,20/,10/}
        \pie [pos={9,0}, radius=2]{30/,25/,15/,10/,10/,5/,5/}
    \end{tikzpicture}
    
    \caption{Generieke portfolio allocatie van links naar rechts: hoog tot lager risico.}
    \label{fig:portfolio allocation}
\end{figure}

\begin{figure}
    \centering
\begin{tikzpicture}
    \pie [pos={0,0}, radius=2, text=pin, rotate=145, explode={0,0,0.2,0.2}]{10/A, 20/B, 30/C, 40/D}
    \pie [pos={9,0}, radius=1.5, rotate=90, color={yellow!60, yellow!40}, explode =0] {60/C1, 40/C2}
    \pie [pos={4.5,0}, radius=1.5, rotate=180, color={orange!40, orange!30, orange!30}, explode =0] {40/D1, 30/D3, 30/D2}
\end{tikzpicture}
    \caption{Generieke representatie van subklassen.}
    \label{fig:subclasses}
\end{figure}


\subsection{Diversificatie van de portefeuille}
Wanneer je diversifieert, probeer je risico's te managen door je beleggingen te spreiden. Je kunt zowel binnen als tussen verschillende activaklassen uitbreiden en diversifi\"eren. Als je diversifieert binnen activaklassen, worden ze aangeduid als subklassen. Dan verdeel je bijvoorbeeld het geld dat je aan \'e\'en bepaalde activaklasse, zoals cryptocurrencies, hebt toegewezen over verschillende categorie\"en van beleggingen die tot die activaklasse behoren. Een belegger kan verschillende grondstoffen bezitten, zoals goud, zilver en palladium, maar tegelijkertijd handelen in olie en koffiebonen. Door tegelijkertijd meerdere subklassen aan te houden die niet gecorreleerd zijn, verminder je effectief het risico. 

    \begin{quotation}
          \textit{\say{Het hoge risicoprofiel van cryptocurrencies maakt het juist interessant. Beleggen in cryptocurrencies kan worden gezien als een asymmetrische weddenschap. In het gunstigste geval kan een kleine inleg resulteren in gigantische winsten en in het slechtste geval in een klein verlies.}}
          \begin{flushright}
            \small{--- \textbf{cryptomanuals}}
          \end{flushright}
    \end{quotation}


\subsection{Herbalanceren van de portefeuille}
Wanneer marktprestaties de waarden van je activaklassen wijzigen, kan het zijn dat je activa-allocatie niet langer het door jou gewenste evenwicht tussen groei en rendement biedt. In dat geval kun je overwegen om je posities aan te passen en je portefeuille te stabiliseren. De activa groeien in verschillende tempo's - wat betekent dat je portefeuille misschien niet meer in lijn is met de door jou gekozen allocatie. Zo kan het zijn dat sommige activa recentelijk in een veel sneller tempo zijn gegroeid. Ter compensatie zou je een deel van de waarde van snelgroeiende activa kunnen herinvesteren in activa met een tragere recente groei, die nu wellicht klaar zijn om stoom af te blazen terwijl de recente 'high-performers' vertragen. Anders zou je kunnen eindigen met een portefeuille die meer risico's met zich meebrengt en een lager lange termijn rendement biedt dan je verwachtte.\medskip 

\subsubsection*{Leren balanceren}
Je kunt een portefeuille op verschillende manieren herverdelen om deze weer in lijn te brengen met jouw geplande allocatie balans. Hier zijn de drie gemeenschappelijke benaderingen om het evenwicht te herstellen:

\begin{enumerate}[label=(\alph*)]
    \item Herverdeel geld naar achterblijvende activaklassen tot ze terugkomen op het percentage van de totale portefeuille dat ze aanhielden in de oorspronkelijke toewijzing.
    \item Voeg nieuwe beleggingen toe aan de achterblijvende activaklassen en concentreer een groter percentage van de bijdragen op die klassen.
    \item Verkoop een deel van jouw bezit in de beleggingscategorie\"en die beter presteren dan andere. Je kunt dan de winst herbeleggen in de achterblijvende activaklassen.
\end{enumerate}

Hoe de portefeuille weer in balans wordt gebracht, hangt volledig af van jouw voorkeuren en de huidige situatie. Over het algemeen voelen mensen zich comfortabeler met de eerste twee alternatieven dan met het derde. Mensen zijn vaak gehecht aan beleggingen die goed presteren en vinden het moeilijk om ze los te laten. Als ze die wel loslaten, zijn ze echter in staat om winst te nemen en die winst te investeren in de ondermaats presterende beleggingen. 

    \bigskip
    \begin{cryptobox}{BIG-CAP MID-CAP \& SMALL-CAP}
            Cryptocurrency projecten verschillen enorm in omvang (de marktkapitalisatie) en worden daarom vaak opgedeeld in de volgende catagorie\"en.
            
            \begin{enumerate}[label=(\alph*)]
            \setlength\itemsep{0em}
                \item \textbf{big-cap}: marktkapitalisatie > \euro 1B 
                \item \textbf{mid-cap}: marktkapitalisatie > \euro 200M
                \item \textbf{small-cap}: marktkapitalisatie < \euro 200M
            \end{enumerate}
            
            Hieronder zijn twee voorbeelden weergegeven van een cryptocurrency portfolio allocatie (afhankelijk van marktkapitalisatie). Check \href{https://coinmarketcap.com/}{CoinMarketCap} of \href{https://www.coingecko.com/en}{CoinGecko} voor de actuele marktkapitalisatie. 
            
            \tcblower
                \begin{tikzpicture}
                    \pie [radius=1.5, text=legend]{50/large-cap, 30/mid-cap, 20/small-cap}
                    \pie [pos={7,0}, radius=1.5, text=legend]{25/large-cap, 50/mid-cap, 25/small-cap}
                \end{tikzpicture}
               
                
    \end{cryptobox}

\section{Activaklassen - soorten beleggingen}
\label{sec:activaklassen}

Een activaklasse is een groep van beleggingen met gelijksoortige kenmerken en financi\"ele structuur. Deze beleggingen worden doorgaans verhandeld op dezelfde financi\"ele markten, met dezelfde regels en voorschriften, en je kunt ze indelen naar locatie. Beleggers en marktanalisten zien beleggingen in binnenlandse effecten, buitenlandse of internationale beleggingen en beleggingen in opkomende markten vaak als verschillende beleggingscategorie\"en. 

\subsection{Cryptocurrencies}
Cryptocurrencies (zoals Bitcoin) zijn een nieuwe, digitale activa klasse die sterk in opkomst is. Een cryptocurrency (of crypto valuta) is een digitaal bezit dat is ontworpen om te fungeren als een ruilmiddel dat gebruikmaakt van sterke cryptografie om financi\"ele transacties te beveiligen, de creatie van extra eenheden te controleren en de overdracht van activa te verifi\"eren. Cryptocurrencies maken gebruik van gedecentraliseerde controle in tegenstelling tot gecentraliseerde digitale valuta en centrale banksystemen. Cryptocurrencies worden vaak beschouwd als een hoog risico vanwege de hoge volatiliteit.

\subsection{Cryptocurrencies - categorie\"en en classificatie}
De markt voor cryptocurrencies is ge\"explodeerd. De cryptocurrencywereld begon met Bitcoin, een gedecentraliseerde peer-to-peer cryptocurrency. Toen cre\"eerde Ethereum een functionele coin, Ether (ETH), dat fungeert als \say{gas} voor een gedecentraliseerd computer besturingssysteem. Vele andere tokens gebouwd op het Ethereum-netwerk volgden. Een dynamische, snelgroeiende markt voor het kopen, verkopen en verhandelen van cryptocurrencies en tokens was het resultaat.
Er zijn nu duizenden tokens, en een groot aantal initial coin offerings [soort crowdfundings], vaak bekend als een \say{ICO}. Hoewel er een aanzienlijke publieke belangstelling en vraag is naar cryptocurrencies en ICO's, is er weinig kennis en structuur voorhanden om tokens te classificeren en te organiseren. De namen die de tokens labelen, zijn meestal organisch en nog niet gestandaardiseerd\footnote{Ethos (2017/2018); \href{https://www.ethos.io/blockchain-finance/}{Ethos Token Classification System: A Framework for Understanding Types of Blockchain Based Tokens}.}. 

    \bigskip
    \begin{cryptobox}{KLASSE 1: UTILITEIT \& FUNCTIONELE TOKENS}
        Om te beginnen zijn er functionele of utilitaire tokens. De waarde van deze tokens is afgeleid van het product waartoe ze toegang geven. Een token dat toegang geeft tot een brug heeft een prijs die afgeleid is van het aantal mensen dat bereid is te betalen om de brug over te steken. De cryptocurrency Ether (ETH) heeft een prijs die wordt afgeleid van het aantal mensen dat bereid is te betalen voor toegang tot het Ethereum Netwerk (gedistribueerd netwerk) waarop alle Ethereum Applications draaien. Dit ecosysteem stelt gebruikers, bedrijven en instellingen in staat om financi\"ele applicaties te cre\"eren die worden aangedreven door open standaarden. Deze tokens staan het dichtst bij virtuele grondstoffen zoals olie, gas, staal, graan of goud.
        \tcblower
        \textbf{Het beste voorbeeld van een utiliteitstoken dat iedereen kent in de praktijk is wel het festivalmuntje. Op het festivalterrein zijn ze van grote waarde, maar daarbuiten heb je er niets aan.}
    \end{cryptobox}

    \medskip
    \begin{cryptobox}{KLASSE 2: TRANSACTIONELE TOKENS (CRYPTOCURRENCIES)}
        Ten tweede zijn er de transactionele tokens of cryptocurrencies. Deze tokens hebben geen inherente waarde, maar ze ontlenen hun waarde aan het netwerkeffect en het vertrouwen dat deze tokens waarde hebben \emph{en behouden}.
        Op dezelfde manier worden traditionele fiat currencies algemeen geaccepteerd als iets dat waarde heeft. Aangezien velen de basis van geld en valuta niet begrijpen, zijn cryptocurrencies vaak nog moeilijker te begrijpen en worden ze daarom beschouwd als bubbels zonder onderliggende (fundamentele) waarde. Afgelopen jaren hebben echter laten zien dat Bitcoin en cryptocurrencies waarde hebben en gebruikt worden voor transacties.
        \tcblower
        \textbf{Transactionele tokens worden vaker als algemeen betaalmiddel gebruikt en niet om een dienst of service te betalen op een (afgebakend) platform, zoals bij de festivalmuntjes het geval is.}
    \end{cryptobox}

    \medskip
    \begin{cryptobox}{KLASSE 3: SECURITY TOKENS OF TOKENIZED SECURITIES}
            Het derde type token is de 'tokenized security'. Deze tokens ontlenen hun waarde aan het feit dat ze iets anders van waarde vertegenwoordigen. Met een tokenized security wordt een claim gelegd op een activa die mogelijk buiten de blockchain ligt en een bepaald vermogen representeert. De meest voorkomende in deze tijd kennen wij als een aandeel in een bedrijf. 
            \tcblower
            \textbf{Je kunt hierbij denken aan het bezitten van 1 gram 'digitaal' goud, in de vorm van 1 token. Je kunt op deze manier fracties van iets bezitten en die verhandelen.}
    \end{cryptobox}

\subsection{Aandelen}
Een aandeel is een deel van het eigendom van een bedrijf. Aandelen worden uitgegeven door beursgenoteerde bedrijven en kunnen worden verhandeld op beurzen (zoals de New York Stock Exchange, NASDAQ, London Stock Exchange, Hong Kong Stock Exchange). Je kunt mogelijk profiteren van aandelen door een stijging van de aandelenprijs of door het ontvangen van een dividend.

\subsection{Onroerend goed en land}
Onroerend goed bestaat uit grond en alles wat zich daarop bevindt, zoals gebouwen, flora en fauna en natuurlijke hulpbronnen zoals gewassen, mineralen en water. Het beleggen in onroerend goed omvat de aankoop, het bezit, het beheer, de verhuur of de verkoop van onroerend goed met winstoogmerk. Vastgoed is een activaklasse met een beperkte liquiditeit in vergelijking met andere investeringen; het is vaak kapitaalintensief en sterk cashflow afhankelijk. Als deze factoren niet goed worden begrepen en beheerd door de investeerder, kan vastgoed een risicovolle investering worden.

\subsection{Grondstoffen}
Grondstoffen zijn basisgoederen die in de handel worden gebruikt en die uitwisselbaar zijn met andere goederen van hetzelfde type.
Enkele van de meest verhandelde grondstoffen zijn goud, rundvlees, hout, olie en aardgas. Andere voorbeelden zijn zilver, koper, ijzererts, zout, suiker, thee, koffie en zo meer. Grondstoffen worden gebruikt als input voor de productie van andere goederen of diensten. De kwaliteit van een bepaalde grondstof kan enigszins verschillen, maar is in principe uniform voor alle producenten. Wanneer deelnemers grondstoffen op een beurs kopen of verkopen, moeten ze ook voldoen aan bepaalde minimumnormen, ook wel basiskwaliteit genoemd.

\subsection{Obligaties en beleggingen met vaste rente}
Obligaties (en andere vaste of variabele rentedragende beleggingen) en schuldbewijzen zoals bedrijfs- of overheidsobligaties keren een rendement uit in de vorm van rente. 

\subsection{Futures en andere financi\"ele derivaten}
Deze categorie omvat futurescontracten, de valutamarkt of Forex (de internationale markt waarop valuta's onderling verhandeld worden), opties en een groeiend aanbod aan financi\"ele derivaten, d.w.z. financi\"ele instrumenten die gebaseerd zijn op of afgeleid zijn van een onderliggend activa. Aandelenopties zijn bijvoorbeeld een derivaat van de onderliggende aandelen.

\subsection{Cash of cashequivalenten}
Het primaire voordeel van cashbeleggingen of cashequivalenten is de liquiditeit. Geld dat wordt aangehouden in de vorm van cashmiddelen of cashequivalenten kan op elk moment snel en gemakkelijk worden opgevraagd en verhandeld. 
\subsection{Overige investeringen}
Het kan hierbij gaan om kunstwerken, diverse andere verzamelobjecten, peer-to-peer lending, hedgefunds, of private investeringen (private equity, venture capital, angel investing). 
Over het algemeen hebben alternatieve beleggingen minder liquiditeit en zijn ze risicovoller. Risico en liquiditeit zijn beide afhankelijk van de positie in de bedrijfscyclus. 

\section{Aandachtspunten}
Tot slot behandelen we enkele essenti\"ele zaken die extra aandacht vereisen bij het investeren, verhandelen en beheren van je beleggings- of handelsportefeuille. Het omvat datgene wat we tot nu toe cruciaal hebben gevonden, en we zullen de lijst blijven aanpassen en verfijnen naarmate we verdergaan en ontwikkelen. We willen benadrukken dat veel dingen uiteindelijk neerkomen op jouw voorkeur, persoonlijkheid en stijl - de manier waarop jij ermee omgaat. Deze informatie dient als een anker voor mensen met beperkte kennis over en ervaring in deze praktijken. Uiteindelijk komt het erop neer dat de investeerders hun eigen onderzoek doen en blijven investeren in hun eigen educatie. Fundamentele analyse wordt beschouwd als een absolute minimumvereiste voor elke investering. Koppel dit aan technische analyse voor een optimaal instapmoment, en je bent goed op weg om een goede investeerder te worden.

\begin{enumerate}[label=(\alph*)]
    \setlength\itemsep{0em}
     \item \textbf{Onderzoek}. Vooral op het gebied van cryptocurrencies zijn er meer projecten dan je kunt tellen en zijn er use cases in bijna elke industri\"ele niche. Er is een verbazingwekkende hoeveelheid investeringsmogelijkheden in de cryptocurrency sector, maar het is uiterst speculatief, tenzij je je bewust bent van het landschap en weet wat je doet. Onbezonnen beslissingen kunnen er gemakkelijk toe leiden dat je jouw investeringen verliest. Kijk dus niet alleen naar de hype en speculatie, maar ga ook na of het project echt nut heeft.
    \item \textbf{Investeer} nooit meer dan je je kunt veroorloven. Investeer alleen op basis van je besteedbare inkomen; je mag nooit gebruikmaken van kredietlijnen of geleend geld.
    \item \textbf{Geduld}. Ga niet te snel shoppen en te veel (niet gediversifieerd) activa aanschaffen als je op lange termijn investeert. Denk in jaren, maanden en kwartalen in plaats van weken en dagen en heb altijd wat reserves klaar liggen voor als de prijzen laag zijn.
    \item \textbf{Asset allocatie}. Spreid de posities niet te dun over te veel activa (d.w.z. twintig verschillende activa met kleine posities). Ook hier is diversificatie de sleutel.
    \item \textbf{Diversificatie}. Probeer verschillende industrie\"en en marktniches te bestrijken. Kijk verder dan cryptocurrencies en kijk ook naar grondstoffen: edelmetalen zoals goud of zilver, aandelen, onroerend goed of grond.
    \item \textbf{Herbalanceren}. Het herevalueren van de portefeuille is een goede oefening en biedt mogelijkheden om te reflecteren op de prestaties en veranderingen binnen het portfolio. Hierna worden posities eventueel (deels) verkocht of aangevuld. 
\end{enumerate}
