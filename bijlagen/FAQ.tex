\chapter{Veelgestelde Vragen}
\label{ch:FAQ}
Hier vind je antwoorden op 10 van de meest gestelde vragen.

\paragraph{Wat is Bitcoin (BTC)?} Bitcoin is 's werelds eerste mens-tot-mens digitale cryptocurrency, oftewel cryptovaluta. Bitcoin, in 2009 uitgebracht door de anonieme Satoshi Nakamoto, is de eerste toepassing van blockchain technologie. In de \href{https://bitcoin.org/bitcoin.pdf}{whitepaper} van Satoshi Nakamoto werd de implementatie ervan uiteengezet: een veilig, wereldwijd, publiek grootboek (ledger) als registratie van elke transactie op het netwerk (de blockchain), onveranderlijk en censuur-bestendig. Bitcoin, open-source software, betekende de eerste applicatie van open finance en is de eerste decentrale cryptocurrency.


\paragraph{Moet ik belasting betalen over mijn Bitcoins?}
Net als een spaarrekening met traditioneel geld of een aandelenportefeuille worden virtuele cryptocurrencies in Nederland gezien als vermogen. Een Bitcoin bezitter moet daarom vermogensbelasting betalen over zijn digitale coins. Sinds vorig jaar staat er bij de aangifte voor box 3 dat 'virtuele betaalmiddelen (bijvoorbeeld Bitcoins)' worden gezien als 'overige bezittingen'.
Over de Bitcoins die u bezit betaalt u op dit moment alleen vermogensbelasting. Een vermogen van minder dan 50 duizend euro is belastingvrij. Winst als gevolg van een koersstijging wordt gezien als vermogensgroei. \medskip
Aangifte. De Belastingdienst is alleen ge{\"i}nteresseerd in de waarde van de cryptomunten op 1 januari. Aan de hand daarvan berekent hij de vermogensbelasting. In april 2021 moeten de belastingbetalers dus doorgeven hoeveel hun cryptomunten op 1 januari 2020 waard waren. Wie in de loop van 2021 Bitcoin heeft gekocht, hoeft dus pas aangifte te doen in 2022.

\paragraph{Hoe werkt Bitcoin?} Om te begrijpen hoe Bitcoin werkt, moeten we eerst begrijpen wat een netwerk is. Een netwerk is in principe een systeem met meerdere nodes (gebruikers) en verbindingen (transacties) tussen deze nodes. Als netwerk is het doel van het Bitcoin-netwerk om gebruikers in staat te stellen om tokens naar elkaar te sturen (Bitcoins). De primaire problemen met transacties in het algemeen zijn die van beveiliging en verantwoording, die beide oplossingen vereisen om fraude te voorkomen. Stel je eens voor dat je een financi{\"e}le transactie doet via het internet waarbij je een bedrag verstuurt. Je hoopt op twee dingen: dat het geld onderweg niet door een derde partij wordt onderschept, en dat je transactie ergens in een grootboek(ledger) wordt vastgelegd als de betaling niet doorgaat. De aanpak van de blockchain om deze problemen op te lossen is wat het tot een revolutionaire technologische innovatie maakt. 

\paragraph{Wat is Blockchain?} De blockchain is een baanbrekende technologie waar veel van de huidige cryptocurrency networken op gebouwd zijn. Op globaal  niveau laat de blockchaintechnologie enkele van de meest nodige oplossingen zien voor netwerkvraagstukken in de menselijke geschiedenis. De gedecentraliseerde controle van elke cryptocurrency werkt door middel van gedistribueerde grootboektechnologie, typisch een blockchain. Een blockchain kan zowel openbaar als priv{\'e} zijn, vergelijkbaar met het intranet versus het internet. De handeling van het inbedden van een vorig blok data in het huidige blok data wordt chaining genoemd, vandaar de naam blockchain.


\paragraph{Wat is Distributed Ledger Technologie?} Een vorm van distributed ledger technologie is het blockchain systeem, dat zowel openbaar als priv{\'e} kan zijn. Een gedistribueerd grootboek (ook wel een gedeeld grootboek of gedistribueerde grootboektechnologie of DLT genoemd) is een consensus van gerepliceerde, gedeelde en gesynchroniseerde digitale gegevens die geografisch verspreid zijn over meerdere locaties, landen of instellingen. Er is geen centrale beheerder of gecentraliseerde gegevensopslag. Er is een peer-to-peer netwerk nodig, evenals consensus algoritmen om ervoor te zorgen dat er geen replicatie tussen de nodes wordt uitgevoerd. 

\paragraph{Wat is Proof of Work (PoW)?} Proof of Work verwijst naar de rekenpuzzel die de miners moeten oplossen, waardoor veel open blockchain netwerken veilig en gedecentraliseerd kunnen blijven. Proof of Work maakt gebruik van cryptografische functies die in garanderen dat er een bepaald aantal computercycli en berekeningen zijn gebruikt om de puzzel op te lossen. Met andere woorden, door deze puzzel op te lossen, bewijst het dat je werk hebt verricht - vandaar de term Proof of Work. 

\paragraph{Wat doen Miners?} Bitcoin is zo ontworpen dat het de hulp nodig heeft van miners: personen of organisaties die hun computer rekenkracht inzetten om de informatie van deze transacties te versleutelen. Miners kunnen grote hoeveelheden hardware gebruiken om cryptografische puzzels op te lossen. Dit gebeurt op een Bitcoin-specifiek computerprogramma dat de informatie van elke transactie op het netwerk bijhoudt. Wanneer een transactie wordt versleuteld door een miner, wordt deze als een nieuw blok toegevoegd aan een verbonden keten van blokken. Dit werkt als een zichtbaar, onveranderbaar openbaar grootboek - vandaar de naam blockchain. In ruil voor al deze computerrekeninspanningen wordt de miner, als de creatie van het nieuwe blok, door het netwerk geldig geacht (via consensus), beloond met een nieuw gecre{\"e}erde Bitcoin. De miner kan deze munt dan naar eigen keuze gebruiken.

\paragraph{Wat is Ethereum (ETH)?} Ethereum is een open platform dat ontwikkelaars in staat stelt om decentrale applicaties, zoals smart contracts en andere complexe juridische en financi\"ele applicaties, te bouwen en in te zetten. Ethereum kan gezien worden als een programmeerbare Bitcoin waar ontwikkelaars de onderliggende blockchain kunnen gebruiken. Om markten, gedeelde grootboeken, digitale organisaties en andere eindeloze mogelijkheden te cre\"eren, die onveranderlijke gegevens en afspraken nodig hebben. Dit allemaal zonder dat er een tussenpersoon nodig is.

\paragraph{Wat is een smart contract?} 
Een smart contract is een toepassing die op een blockchain netwerk draait. Smart contracts die op openbare blockchain netwerken worden ingezet, zijn zelfuitvoerbaar en onveranderlijk na ondertekening. Het gebruik van dergelijke contracten is grenzeloos, omdat ze kunnen worden gebruikt om gedecentraliseerde uitwisselingen, tokenized activa, spelletjes en meer te bouwen. Sinds het eerste smart contract platform, Ethereum, in 2015 werd uitgebracht, zijn smart contracts het primaire aandachtspunt geworden van innovatoren in de blockchain wereld.

\paragraph{Wat zijn cryptocurrencies?} Een cryptocurrency (of crypto valuta) is een digitaal bezit dat is ontworpen om te werken als een ruilmiddel dat gebruik maakt van sterke cryptografie om financi\"ele transacties te beveiligen, de creatie van extra eenheden te controleren en de overdracht van activa te verifi\"eren. Cryptocurrencies maken gebruik van gedecentraliseerde controle in tegenstelling tot gecentraliseerde digitale valuta en centrale banksystemen.

\paragraph{Wat is een Peer to Peer Netwerk?} P2P staat voor \say{Peer to Peer,}. In een P2P-netwerk zijn de \say{peers} computersystemen die via het internet met elkaar verbonden zijn. P2P-netwerken worden meestal gevormd door groepen computers. Deze computers slaan allemaal hun gegevens op met behulp van individuele beveiliging, maar delen ook gegevens met alle andere nodes (knooppunten). Bestanden kunnen direct worden gedeeld tussen systemen op het netwerk zonder dat er een centrale server nodig is.Een peer-to-peer (P2P) dienst is een decentraal platform waarbij twee (of meer) personen direct met elkaar in contact komen, zonder tussenkomst van een derde partij. In plaats daarvan handelen de koper en de verkoper rechtstreeks met elkaar via de P2P-service. 

%\paragraph{What is Proof of Stake (PoS)?} The Proof of Stake (PoS) was created as an alternative to the Proof of Work (PoW), to tackle inherent issues like energy usage. Proof of Stake concept states that a person can mine or validate block transactions according to how many coins he or she holds. The proof of stake (PoS) seeks to address this issue by attributing mining power to the proportion of coins held by a miner. This way, instead of utilizing energy to answer PoW puzzles, a PoS miner is limited to mining a percentage of transactions that is reflective of his or her ownership stake. For instance, a miner who owns 3\% of the Bitcoin available can theoretically mine only 3\% of the blocks.

%\paragraph{How do cryptocurrency wallets work?} Unlike traditional [pocket] wallets, digital wallets do not store any currency. Cryptocurrencies do not get stored in any single location or exist anywhere in any physical form. All there is are records of transactions stored on the blockchain.

%\paragraph{What are cryptocurrency wallets?} Cryptocurrency wallets are software programs that work with public and private key pairs. The wallets provide a user interface via which you can monitor their recorded balance on the network and conduct transactions. When a person sends you Bitcoin or any other type of digital currency, they are permanently signing off ownership of the coins linked to their wallet address, and once the transaction is verified, the network records Bitcoins ownership on your address.

%\paragraph{Recovery or Seed phrases} Anytime you set up a wallet; users are provided with a unique recovery seed composed of anywhere from 12 to 24 randomized words (12 and 24 being most common). You are encouraged to write this recovery seed down somewhere safe and never to post it online. Recovery seeds are considered the most crucial aspect of maintaining the safety of your cryptocurrencies. A recovery seed is your best friend when you lose your paper, hardware, or mobile wallet, as its the only way you can recover your funds and wallet. In summary, if you ever lose access to your wallet for whatever reasons, the seed phrase is what you require to recover your lost funds. 

%\paragraph{Cryptography} The field of cryptography is fundamental to many cryptocurrencies such as Bitcoin. Cryptography is the practice of secure communication in the presence of third parties. In other words, cryptography allows for data to be stored and communicated in such a way that third parties are prevented from reading the contents. Cryptography is utilized in the creation of public and private keys to make cryptocurrency systems a secure network upon which users can safely operate.

%\paragraph{Ownership on a cryptocurrency} The concept of ownership on a cryptocurrency system is primarily comprised of three interconnected elements:

%\item{Public and private keys}
%\item{Public addresses (hashed public key)}
%\item{Digital signatures}

%\paragraph{What are public and private keys?} When dealing with cryptocurrency, users are given a public address and a private key to send and receive coins or tokens. The public address is where the funds are deposited and received. However, even though a user has tokens deposited into his address, he won\'t be able to withdraw them without the unique private key. The public address is generated by hashing the public key. In turn, the public key is created from the private key through a complicated mathematical algorithm (hashing). However, it is near impossible to reverse the process by generating a private key from a public key. 


%\paragraph{What is a cryptocurrency public address} Your public address is essentially a secured cryptographic representation of your public key. Think of this address as having a mailbox with a specific string of numbers attached as an identifier and only you can access it with your private key. Usually, these public addresses will have an automated copy button available and will also be shown as a QR code so you can easily scan it with a mobile device. Remember that for each public address you have both a public and a private key that match it!


%\paragraph{Digital Signatures} Digital signatures play an important role in cryptocurrency systems because they prove ownership of funds and allow the individual in control of those funds to spend them. With Bitcoin, a digital signature is effectively intended to serve three distinct purposes:
%\begin{enumerate}
%\item{A digital signature serves as proof that the owner of a private key, who will by extension have ownership of his/her funds, has indeed authorized that those funds can be spent.}
%\item{A digital signature serves as proof that the authorization is undeniable.}
%\item{digital signature proves that the transaction that has been authorized by the signature has not or cannot be modified by anyone after it has been signed.}
%\end{enumerate}

 
 
%\paragraph{Transactions - Private Keys} You can safely send your coins from any exchange to any wallet by withdrawing from one address and sending to another address. For instance, you can own multiple wallets (one on your pc, one on your mobile and a hardware wallet) which provide you with at least the same amount of private keys, public keys and public addresses. You need your private key to sign off on transactions. Hence it would be best if you backed it up somewhere safe. If people somehow obtain your private key, they will be able to access your funds, and your wallet is compromised.


%\paragraph{Transact ownership rights} When you want to send a transaction to another public address, you are transferring ownership rights of assets linked to your wallets public address to that of another public address, which is now only accessible by the corresponding private key of that specific person. Thus, if you somehow lose access to your private key, you will be unable to access your funds. Always make sure to backup your private keys or passwords that you have set up in multiple safe locations.
 
%\paragraph{Hashing versus Encryption} Hashing is great for usage in any instance where you want to compare a value with a stored value, but can't store its plain representation for security reasons. Other use cases could be checking the last few digits of a credit card match up with user input or comparing the hash of a file you have with the hash of it stored in a database to make sure that they\'re both the same.
 
%\paragraph{Hashing} A hash is a string or number generated from a string of text. The resulting string or number is a fixed length and will vary widely with small variations in input. The best hashing algorithms are designed so that it\'s impossible to turn a hash back into its original string.

%\paragraph{Encryption} Encryption turns data into a series of unreadable characters, that aren't of a fixed length. The key difference between encryption and hashing is that encrypted strings can be reversed back into their original decrypted form if you have the right key. Encryption should only ever be used over hashing when it is a necessity to decrypt the resulting message. For example, if you were trying to send secure messages to someone on the other side of the world, you would need to use encryption rather than hashing, as the message is no use to the receiver if they cannot decrypt it. 

%\section{Crypto Slang} Bitcointalk forums, Facebook groups, Reddit, Telegram and many other platforms all provide opportunities for crypto enthusiasts to discuss the latest news events, to exchange investment strategies, and to help each other out with some of the more practical aspects of owning, storing and trading cryptocurrency. If you want to make full use of the online communities resources, a wealth of crypto (and often cryptic!) jargon has now become commonplace. Thus, it makes sense that you get to grips with the meaning of such terms when you encounter them. Crypto slang is colorful, unusual, occasionally indecipherable, and - to more experienced cryptocurrency enthusiasts and traders - often the sign of a wide-eyed newbie. From Lambos to Shitcoins (and rarely the other way around) we present a glossary of the terminology you'll encounter in the crypto-world.

%\paragraph{Altcoin} All coins that originated after Bitcoin. Altcoins are the alternative cryptocurrencies launched after the success of Bitcoin. Generally, they project themselves as better substitutes to Bitcoin. The success of Bitcoin as the first peer-to-peer digital currency paved the way for many to follow. Many altcoins are trying to target any perceived limitations that Bitcoin has and come up with newer versions with competitive advantages. As the term 'altcoins' means all cryptocurrencies which are not Bitcoin, there are hundreds of altcoins. 

%\paragraph{DYOR} Do Your Own Research. This stresses that you should always investigate yourself and should not rely on others when making your own decisions. There are a lot of people asking advice and there are also many people shilling coins or projects. Make sure you know what you are getting into.

%\paragraph{FOMO} Fear Of Missing Out. Market panic tends to get the better of people who just entered the market. When the market sentiment is largely positive and prices are on the rise, people get this urge that they are missing out on huge returns and start buying at unrealistic and unsustainable price ranges. When the hype is over and people start to see that this positive surge can't sustain itself, panic enters the market and people start selling quickly. This leads to a lot of people getting burned in the end. Always be watchful when the markets are on a rampage.

%\paragraph{FUD} Fear, Uncertainty \& Doubt. People who are spreading FUD are mostly people who are spreading rumors or nonsense based on nothing but their own opinion or experience. Everybody has an opinion and it's very easy to lose heart if you have lost a significant portion of an investment. Media also spreads FUD occasionally.

%\paragraph{HODL} Hold On for Dear Life. Whatever you do, don't sell your coins. Hold them throughout both good and bad times. This term originated in a bitcointalk forum post where somebody misspelled "hold" and coined the term "hodl". It proceeded to becoming legendary.

%\paragraph{Moon} Prices or coins that are going to the moon (when moon) indicate a dramatic surge in prices. The moon bit almost never happens though and a lot of people end up losing money. Be careful when you notice a lot of people drooling over prices.

%\paragraph{Lambo} Lambo stands for Lamborghini, which is a very expensive sports car. "When Lambo" practically means the same as "When Moon" since they both refer to anticipated price surges where people expect to make so much money that they can buy a Lamborghini. It does occasionally happen that someone is able to make huge returns but usually a lot of other people get burned in the process.

%\paragraph{Shitcoin} Shitcoin is a pejorative term used to describe an altcoin that has become worthless. Shitcoin value may disappear because interest failed to materialize, because the altcoin itself was not created in good faith, or because the price was based on speculation.  

%\paragraph{BTD or BTFD} Buy The (Fucking) Dip. Buying the dips occurs after there has been a significant dip in the price of a security or stock index. Investors practicing this method will increase positions or purchase these newly lower-priced stocks to capitalize on what they hope will be a coming upswing in prices.  

%\paragraph{Bag holder} An informal term used to describe an investor who holds a position in a security which decreases in value until it is worthless. In most cases, the bag holder will hold the position for an extended period during which most of the investment is lost.  

%\paragraph{Pump} A sudden surge in prices of a crypto, stock or other type of security. Could be part of a pump and dump scheme, or based on actual developments or value being created such as new releases, alpha and beta stages etc.

%\paragraph{Dump} A drop in prices which could be caused by a major set back within the project itself, pump and dump schemes or investors selling stocks at high prices in order to take profits.

%\paragraph{Pump \& Dump} Pump and dump is a scheme that attempts to boost the price of a stock through recommendations based on false, misleading or greatly exaggerated statements. The perpetrators of this scheme, who already have an established position in the company's stock, sell their positions after the hype has led to a higher share price. This practice is illegal based on securities law and can lead to heavy fines.  

%\paragraph{Whale} A whale is term in the cryptocurrency world used to refer to individuals or entities that hold large amounts of coins or tokens. From the point of view of blockchain and its core decentralized feature, whales cause concern, as the situation could lead to a small number of people having controlling power over the cryptocurrency and if they work together might be able to manipulate prices and markets.  

%\paragraph{Bear/Bearish} A bear is an investor who believes that a particular security or market is headed downward and attempts to profit from a decline in stock prices. Bears are typically pessimistic about the state of a given market. For example, if an investor were bearish on the Standard \& Poor's (S\&P) 500, that investor would attempt to profit from a decline in the broad market index.  

%\paragraph{Bear trap} A Bear Trap is a technical pattern that occurs when the performance of a stock or an index incorrectly signals a reversal of a rising price trend.  

%\paragraph{Bull/Bullish} A bull is an investor who thinks the market, a specific security or an industry is poised to rise. Investors who adopt a bull approach purchase securities under the assumption that they can sell them later at a higher price. Bulls are optimistic investors who are attempting to profit from the upward movement of stocks.  

%\paragraph{Bull trap} A bull trap is a false signal indicating that a declining trend in a stock or index has reversed and is heading upwards when, in fact, the security will continue to decline. The move "traps" traders or investors that acted on the buy signal and generates losses on resulting long positions. A bull trap may also be referred to as a whipsaw pattern.  

%\paragraph{Fundamental Analysis} Fundamental analysis [FA] instead looks at economic and financial factors that influence a business. Fundamental analysis is the cornerstone of investing. In fact, some would say that you aren't really investing if you aren't performing fundamental analysis. Because the subject is so broad, however, it's tough to know where to start. There are an endless number of investment strategies that are very different from each other, yet almost all use the fundamentals. Within the cryptocurrency market, the process differs slightly when compared to investing in regular companies and corporations. Please refer to our own section on performing fundamental analysis for cryptocurrency related projects.

%\paragraph{Technical Analysis} Technical analysts [TA] typically begin their analysis with charts because technical analysis looks at the price movement of a security and uses this data to predict future price movements. Technical analysts believe that there's no reason to analyze a company's financial statements since the stock price already includes all relevant information. Instead, the analyst focuses on analyzing the stock chart itself for hints into where the price may be headed.  

%\paragraph{Faucet} Bitcoin faucets are reward systems that are run on exclusive websites, portals, or digital apps. They are used to reward users by paying them small amounts of satoshi in exchange for completing a task. A bitcoin faucet allows one to earn satoshi, which is a fraction of a bitcoin, either by performing a simple task - like solving a captcha puzzle or completing surveys - or by visiting an advertiser or partner website for a specified period of time. In addition to bitcoin, there are faucets for other cryptocurrencies as well.   